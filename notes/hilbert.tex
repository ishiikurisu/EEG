\documentclass[12pt]{article}

\begin{document}

The Hilbert transform $H$ of a signal $u$ is defined as
$$H(u)(t):= (u(s) \star \frac{1}{\pi s})(t)$$
$$= p.v. \int_{-\infty}^{\infty} u(s) h(t-s) ds$$
$$= \frac{1}{\pi} p.v. \int_{-\infty}^{\infty} \frac{u(s)}{t-s} ds$$
Property:
$$H(H(u))(t) = -u(t)$$
Relationship to the Fourier transform:
$$F(H(u))(w)=(-\jmath \cdot sgn(w)) F(u)(w)$$
where
$$F(u)(w):=\frac{1}{2\pi} \int_{-\infty}^{+\infty} u(t)e^{-\jmath w t} dt$$
is the Fourier transform of the signal $u$ and
$$sgn(w):=\frac{w}{\Arrowvert w \Arrowvert}$$
is the signal of $w \in \Im$. Therefore, my conclusion is

$$H(u)(t) = F^{-1}(w \mapsto -\jmath \cdot sgn(w) \cdot F(u)(w))(t)$$

\end{document}
